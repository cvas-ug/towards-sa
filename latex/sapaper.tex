\documentclass[12pt]{article}

\begin{document}
\title{Robotic Self-awareness}
\maketitle

\begin{center}
    \vspace{0.4cm}
    \textbf{Authors Names}
    
    \vspace{0.9cm}
    \textbf{Abstract}
\end{center}
This paper presents the first self-awareness experiments.
\pagebreak

\section{Introduction}
\begin{itemize}
\item Small story behined this paper
\end{itemize}
With current advances in the robotics multisensory platforms, it is time to utilise these capabilities and construct a robot that able to dynamically interact with humans and other robotic agents.
\begin{itemize}
\item last paragraph of the paper structure.
\end{itemize}
\subsection{What is self-Awareness}
\begin{itemize}
\item self-awareness difinition in human.
\item why it is important in the human (one sentance).
\item self-aware the robot fesaibility, if applicable.
\item emphsis the in-to-out approch (one sentence).
\end{itemize}
\subsection{Absence of self-awareness and the problems associated with a robot lack of self-awareness}
\begin{itemize}
\item current robot status and how robots works in some different environments.
\item some problems because of robot is not aware.
\item possible applications.
\end{itemize}
\section{Literature review}
\subsection{levels of self-awareness}
\begin{itemize}
\item Human five levels of self-awareness
\end{itemize}
\subsection{Related literature on self-awareness}
\begin{itemize}
\item current efforts to create the robot self-awareness.
\item other methods and approches to create the Robot self-aware.
\item our method and approch to create the robot self-awareness.
\item why we see our approch is better and unique (from in-to-out).
\end{itemize}

\section{The Self-Aware Architecture}
\subsection{Design and Rationale}
\subsection{Materials and Methods}
\subsection{Implementation of Level 1}

\section{Experiment}
\subsection{Evaluation cases}
\subsection{Discussion}
\subsubsection{How to move forward to Level 2}


\section{Conclusions and Future Work}
Work on next level of self-awareness.

\end{document}