\documentclass[12pt]{article}

\begin{document}
\title{Robotic Self-awareness}
\maketitle

\begin{center}
    \vspace{0.4cm}
    \textbf{Authors Names}
    
    \vspace{0.9cm}
    \textbf{Abstract}
\end{center}
This paper presents the first self-awareness experiment for a robot. The robot got to know it self before deal with the environemnt. 
\pagebreak

\section{Introduction}
\begin{itemize}
\item Small story behined this paper
\end{itemize}
With current advances in the robotics multisensory platforms, it is time to utilise these capabilities and construct a robot that able to dynamically interact with humans and other robotic agents.

Our approach to start with implementing the priminalary part of self-awareness which is mostly missing by others. The agent try to interact with itself to construct a self recognition before interacting and deal with the environment objects. the sense of self is constructed from comprehend both spatial and internal sensory. 
 
\subsection{What is self-Awareness}
\begin{itemize}
\item self-awareness difinition in human.
\item why it is important in the human (one sentance).
\item self-aware the robot fesaibility, if applicable.
\newline
Question: Why it is important to know your self before doing the task? Most or all  current tasks are done without referal or preior recognition.
\item emphsis the in-to-out approch (one sentence).
\end{itemize}
\subsection{Absence of self-awareness and the problems associated with a robot lack of self-awareness}
\begin{itemize}
\item current robot status and how robots works in some different environments.
\end{itemize}
\begin{itemize}
\item some problems because of robot is not aware.
\end{itemize}
A basic hand out task is happining between two parties, one is the robot itself and a second robot's hand. If the robot not able to know which of two hands are belong to itself, the task can not be established autonomusly. The trajectory calculation might not be corresponds to the corrct path, from its hand to the target.

\begin{itemize}
\item possible applications.
\end{itemize}
The robot can infer its body as a distinct entity within other world's entities which give more potential to avoid obstacles and source of collisions.

\begin{itemize}
\item list paragraphs to indicates the paper structure.
\end{itemize}

\section{Literature review}
\subsection{levels of self-awareness}
\begin{itemize}
\item Human five levels of self-awareness
\end{itemize}
\subsection{Related literature on self-awareness}
\begin{itemize}
\item current efforts to create the robot self-awareness.
\item other methods and approches to create the Robot self-aware.
\item our method and approch to create the robot self-awareness.
\item why we see our approch is better and unique (from in-to-out).
\end{itemize}

\section{The Self-Aware Architecture}
\subsection{Design and Rationale}
\begin{itemize}
\item design to devise SA in robot.
\item utilise the sensories in relation with from in-to-out rationale.
\item the design of DNN model (selfy model). 
\item maybe other rationals behind the design.
\end{itemize}
\subsection{Materials and Methods}
\begin{itemize}
\item required data and the explination of that data.
\item data collection and the data description.
\item maybe how data collected.
\end{itemize}
\subsection{Implementation of Level 1}
\begin{itemize}
\item process description for DNN model with collected data.
\end{itemize}
\section{Experiment}
\subsection{Evaluation cases}
\begin{itemize}
\item descripe the cases.
\item results achived.
\end{itemize}
\subsection{Discussion}
\subsubsection{How to move forward to Level 2}
knowing the self-body is achived in this expirement and that constructed the initial sense of self in the baxter robot, but the body atriculation and exact situation is not figured out yet by baxter, thus develping situation stage wich represent level 2 of self-awareness is the next step to let robot able to control its self in the environment.

\section{Conclusions and Future Work}
Work on next level of self-awareness.

\end{document}